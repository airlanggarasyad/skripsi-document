\textit{The Trans Gadjah Mada Electric Bus is one of the facilities provided by Universitas Gadjah Mada to facilitate intra-campus academic mobility. This research aims to evaluate the performance of the GNSS Teseo\hyp{}LIV3FL module and develop firmware for tracking the Trans Gadjah Mada Bus using the STM32 platform.}

\textit{Performance evaluation of the GNSS Teseo\hyp{}LIV3FL module is conducted by configuring the module to receive signals from various constellations or multi-constellations. The GNSS constellations used include GPS, BeiDou, Galileo, and QZSS. The test results in each scenario show that the best scenario for the Teseo\hyp{}LIV3FL module is in an open space, indicated by the HDOP value within the ideal range and the VDOP and PDOP values within the excellent range. In addition to testing the multi-constellation performance, this research also examines the performance of the low-power mode algorithm on the GNSS module. The low-power mode algorithm on the Teseo\hyp{}LIV3FL module allows the module to operate at a current of 15 $\mu A$ in standby mode, with a maximum standby time of 5 minutes before the GNSS module finds it difficult to obtain a fix.}

\textit{The developed firmware is intended to process NMEA sentences sent by the GNSS Teseo\hyp{}LIV3FL module into a more readable format. Additionally, the developed firmware also includes geofencing features. The geofencing feature is used to determine whether the current position of the bus is inside or outside the Universitas Gadjah Mada campus environment. The developed firmware has been tested directly on Route 1B of the Trans Gadjah Mada Bus. Based on the conducted tests, the system is able to determine the bus position accurately, and both geofencing features are functioning properly.}

\noindent\textbf{Keywords} : STM32, Teseo\hyp{}LIV3FL, firmware development, position tracking, low power mode algorithm, multi-constellation