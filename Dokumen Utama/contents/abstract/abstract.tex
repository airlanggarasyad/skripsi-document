The Trans Gadjah Mada Electric Bus is one of the facilities provided by Gadjah Mada University to facilitate intracampus academic mobility. This research aims to evaluate the performance of the GNSS Teseo-LIV3FL module and develop a Bus Trans Gadjah Mada tracker firmware using the STM32 platform.

Performance evaluation of the GNSS Teseo-LIV3FL module is conducted by setting the module to receive signals from various constellations. The GNSS constellations used are GPS, BeiDou, Galileo, and QZSS. In addition to testing the multi-constellation performance, this study will also test the low power mode algorithm performance of the GNSS module. There are two low power mode algorithms provided by the GNSS Teseo-LIV3FL module, namely periodic and cyclic. The low power mode algorithm used in this study is the periodic algorithm.

The developed firmware is intended to process NMEA sentences sent by the GNSS Teseo-LIV3FL module into an easier-to-read format. In addition, the developed firmware also has a geofencing feature. The geofencing feature is used to determine whether the current bus position is within the Gadjah Mada University campus area or not. The developed firmware will be directly tested on Route 1B of Trans Gadjah Mada.


\noindent\textbf{Keywords} : STM32, Teseo-LIV3FL, firmware development, position tracking, low power mode algorithm, multi-constellation