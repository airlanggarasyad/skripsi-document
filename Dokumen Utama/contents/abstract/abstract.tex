The Trans Gadjah Mada Electric Bus is one of the facilities provided by Universitas Gadjah Mada to facilitate intracampus academic mobility. This research aims to evaluate the performance of the Teseo-LIV3FL GNSS module and develop firmware for tracking the Trans Gadjah Mada Bus using the STM32 platform.

The performance evaluation of the Teseo-LIV3FL GNSS module is conducted by configuring the module to receive signals from various constellations or multi-constellation. The GNSS constellations used are GPS, BeiDou, Galileo, and QZSS. The test results in each scenario show that the best scenario for the Teseo-LIV3FL module is in an open space, indicated by the HDOP value within the ideal range, and the VDOP and PDOP values within the excellent range. In addition to testing the multi-constellation performance, this research will also evaluate the low-power mode algorithm of the GNSS module. The low-power mode algorithm in the Teseo-LIV3FL module allows it to operate at a standby current of 65mA.

The developed firmware is intended to process the NMEA sentences sent by the Teseo-LIV3FL GNSS module into a more readable format. Additionally, the developed firmware also includes a geofencing feature. The geofencing feature is used to determine whether the current bus position is within the Universitas Gadjah Mada campus environment or not. The developed firmware will be directly tested on Route 1B of the Trans Gadjah Mada Bus. Based on the conducted tests, the system is able to accurately determine the bus position, and both geofencing features are functioning properly.

\noindent\textbf{Keywords} : STM32, Teseo-LIV3FL, firmware development, position tracking, low power mode algorithm, multi-constellation