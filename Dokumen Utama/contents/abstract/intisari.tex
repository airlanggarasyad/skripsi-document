Bus Listrik Trans Gadjah Mada merupakan salah satu fasilitas yang diberikan oleh Universitas Gadjah Mada untuk mempermudah mobilisasi sivitas akademia intrakampus. Penelitian ini memiliki tujuan untuk mengevaluasi performa dari modul GNSS Teseo-LIV3FL dan mengembangkan \textit{firmware} pelacak Bus Trans Gadjah Mada menggunakan \textit{platform} STM32.

Evaluasi performa modul GNSS Teseo-LIV3FL dilakukan dengan mengatur modul untuk menerima isyarat dari berbagai konstelasi atau \textit{multi-constellation}. Adapun konstelasi GNSS yang digunakan adalah GPS, BeiDou, Galileo, dan QZSS. Selain menguji performa \textit{multi-constellation}, pada penelitian ini juga akan diuji performa algoritma mode daya rendah pada modul GNSS. Algoritma daya rendah yang akan digunakan adalah algoritma periodik.

\textit{Firmware} yang dikembangkan ditujukan untuk mengolah kalimat NMEA yang dikirimkan oleh modul GNSS Teseo-LIV3FL menjadi bentuk yang lebih mudah dibaca. Selain itu, \textit{firmware} yang dikembangkan juga memiliki fitur \textit{geofencing}. Fitur \textit{geofencing} yang dimaksud adalah untuk menentukan apakah posisi bus saat ini berada di dalam lingkungan kampus Universitas Gadjah Mada atau tidak. \textit{Firmware} yang dikembangkan akan diujikan secara langsung pada Rute 1B Trans Gadjah Mada.

\noindent\textbf{Kata kunci} -- STM32, Teseo-LIV3FL, pengembangan \textit{firmware}, pelacakan posisi, algoritma daya rendah, \textit{multi-cosntellation}