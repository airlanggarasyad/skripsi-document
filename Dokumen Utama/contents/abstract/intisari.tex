Bus Listrik Trans Gadjah Mada merupakan salah satu fasilitas yang diberikan oleh Universitas Gadjah Mada untuk mempermudah mobilisasi civitas akademika intra kampus. Penelitian ini memiliki tujuan untuk mengevaluasi performa dari modul GNSS Teseo\hyp{}LIV3FL dan mengembangkan \textit{firmware} pelacak bus Trans Gadjah Mada menggunakan \textit{platform} STM32.

Evaluasi performa modul GNSS Teseo\hyp{}LIV3FL dilakukan dengan mengatur modul untuk menerima isyarat dari berbagai konstelasi atau \textit{multi-constellation}. Adapun konstelasi GNSS yang digunakan adalah GPS, BeiDou, Galileo, dan QZSS. Hasil pengujian di setiap skenario menunjukan bahwa skenario terbaik untuk modul Teseo\hyp{}LIV3FL adalah pada ruang terbuka yang ditunjukan dengan nilai HDOP berada dalam rentang ideal dan nilai VDOP dan PDOP berada dalam rentang sangat baik. Selain menguji performa \textit{multi-constellation}, pada penelitian ini juga akan diuji performa algoritma mode daya rendah pada modul GNSS. Algoritma mode daya rendah pada modul Teseo\hyp{}LIV3FL memungkinkan modul untuk beroperasi pada arus 15 $\mu A$ dalam keadaan \textit{standby} dengan waktu \textit{standby} maksimum sebelum modul GNSS sulit untuk mendapatkan fiksasi adalah selama 5 menit.

\textit{Firmware} yang dikembangkan ditujukan untuk mengolah kalimat NMEA yang dikirimkan oleh modul GNSS Teseo\hyp{}LIV3FL menjadi bentuk yang lebih mudah dibaca. Selain itu, \textit{firmware} yang dikembangkan juga memiliki fitur \textit{geofencing}. Fitur \textit{geofencing} yang dimaksud adalah untuk menentukan apakah posisi bus saat ini berada di dalam lingkungan kampus Universitas Gadjah Mada atau tidak. \textit{Firmware} yang dikembangkan akan diujikan secara langsung pada rute 1B Trans Gadjah Mada. Berdasarkan pengujian yang dilakukan, sistem sudah dapat menentukan posisi bus dengan baik dan kedua fitur \textit{geofencing} juga sudah berjalan dengan baik.

\noindent\textbf{Kata kunci} -- STM32, Teseo\hyp{}LIV3FL, pengembangan \textit{firmware}, pelacakan posisi, algoritma daya rendah, \textit{multi-cosntellation}