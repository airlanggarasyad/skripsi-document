\chapter{KESIMPULAN DAN SARAN}

\section{Kesimpulan}
Berdasarkan penelitian yang telah dilakukan, dapat ditarik beberapa kesimpulan sebagai berikut:
\begin{enumerate}
	\item Penggunaan \textit{multi-constellation} pada modul Teseo\hyp{}LIV3FL dapat membantu modul untuk mendapatkan fiksasi di berbagai macam skenario. Adapun hasil pengujian \textit{Rapid Static Survey} untuk setiap skenario adalah sebagai berikut:
	\begin{enumerate}
		\item Skenario \textit{basement} menunjukkan hasil paling buruk jika dibandingkan dengan tiga skenario lainnya. Pada skenario ini, nilai MAD dari modul GNSS adalah sebesar 24,11 meter; nilai CEP rata-rata 32,69 meter; nilai HDOP rata-rata sebesar 8,27; nilai PDOP rata-rata sebesar 10,67; nilai VDOP rata-rata sebesar 8,27; dan visibilitas satelit rata-rata sebanyak 7,60 satelit.
		
		\item Skenario dalam ruangan menunjukan hasil yang sedikit lebih baik jika dibandingkan dengan skenario sebelumnya, yaitu dengan nilai MAD 8,46 meter; ; nilai CEP rata-rata 12,14 meter; nilai HDOP rata-rata sebesar 2,79; nilai VDOP rata-rata sebesar 2,48; nilai PDOP rata-rata sebesar 3,73; dan visibilitas satelit rata-rata sebanyak 10,93 satelit.
		
		\item Skenario ruangan semi terbuka kembali menunjukan hasil yang lebih baik dibandingkan dua skenario sebelumnya. Adapun hasil yang didapat adalah nilai MAD sebesar 7,60 meter; ; nilai CEP rata-rata 7,60 meter; nilai HDOP rata-rata sebesar 0,83; nilai VDOP rata-rata sebesar 1,35; nilai PDOP rata-rata sebesar 1,58; dan visibilitas satelit rata-rata sebanyak 16,68 satelit.
		
		\item Skenario ruangan terbuka menunjukan hasil paling baik dari seluruh skenario yang diuji, yaitu dengan niai MAD 1,21 meter; ; nilai CEP rata-rata 6,12 meter; nilai HDOP rata-rata sebesar 0,65; nilai VDOP rata-rata sebesar 1,30; nilai PDOP rata-rata sebesar 1,12; dan visibilitas satelit rata-rata sebanyak 21,14 satelit.
	\end{enumerate}
	
	\item Algoritma mode daya rendah pada modul Teseo\hyp{}LIV3FL memungkinkan modul untuk bekerja dengan daya yang lebih rendah. Arus yang mengalir pada modul adalah sebesar 10 $\mu$A pada mode \textit{standby} dan 50mA pada mode akuisisi dengan waktu \textit{standby} paling ideal adalah kurang dari 6 menit sebelum modul Teseo\hyp{}LIV3FL sulit untuk mendapat fiksasi kembali.
	
	\item Berdasarkan pengujian secara langsung, hasil pelacakan posisi bus menggunakan \textit{firmware} yang dikembangkan telah mendekati rute yang dipublikasikan oleh Direktorat Pengelolaan dan Pemeliharaan Aset Universitas Gadjah Mada. Kualitas hasil pelacakan juga sudah sangat baik ditunjukan dengan nilai rata-rata dari visibilitas satelit sebesar 9,98; HDOP sebesar 1,12; VDOP sebesar 1,72; dan PDOP sebesar 1,98.
\end{enumerate}
\section{Saran}
Adapun saran untuk penelitian lebih lanjut adalah sebagai berikut:
\begin{enumerate}
	\item Perlunya penelitian lebih lanjut mengenai performa modul Teseo\hyp{}LIV3FL jika dilakukan pengamatan dengan berbagai kombinasi dua atau tiga konstelasi secara bersamaan.
	
	\item Perlunya penelitian lebih lanjut untuk pengembangan aplikasi dasbor visualisasi data atau integrasi dengan aplikasi yang sudah ada.
	
	\item Perlunya penelitian lebih lanjut untuk transmisi data secara nirkabel.
\end{enumerate}

