Puji syukur ke hadirat Allah SWT yang telah melimpahkan rahmat dan barokah-Nya sehingga penulis dapat menyelesaikan skripsi dengan judul Pengembangan \textit{Firmware Tracker} Bus Kampus dengan Modul GPS pada Platform STM32. Laporan skripsi ini disusun untuk memenuhi salah satu syarat dalam memperoleh gelar Sarjana Teknik (S.T.) pada Program Studi S1 Teknik Elektro Fakultas Teknik Universitas Gadjah Mada Yogyakarta.


Dalam melakukan penelitian dan penyusunan laporan skripsi ini, Penulis telah mendapatkan banyak dukungan dan bantuan dari berbagai pihak. Penulis mengucapkan terima kasih yang tak terhingga kepada:

\begin{enumerate}
	\item Bapak Antony Ramdhan, S.E., M.Ak., Ak., CA dan Ibu Dr. Dini Wahjoe Hapsari, S.E., M.Si., Ak., CA selaku kedua orang tua penulis yang telah memotivasi, mendoakan, mendidik, dan juga membayar UKT untuk studi penulis
	
	\item M. Andika Naufan Dwiandharu dan Alvyno Fauzan Tristananda yang senantiasa mendukung dan mendoakan selama melakukan studi.
	
	\item Bapak Anugrahadi Wijayanto dan Ibu Suastutiningsih, S.H. selaku keluarga penulis  yang telah mengizinkan untuk tinggal bersama selama menjalani studi.
	
	\item Rocky dan Luna yang telah ikhlas menjadi pelampiasan penat dengan bayaran sebungkus Friskies Indoor Delight.
	
	\item Ir. Hanung Adi Nugroho, S.T., M.E., Ph.D., IPM. selaku Ketua Departemen Teknik Elektro dan Teknologi Informasi dan Ir. Adha Imam Cahyadi, S.T., M.Eng., D.Eng., IPM. selaku Ketua Program Studi S1 Teknik Elektro Fakultas Teknik Universitas Gadjah Mada yang memberikan izin kepada penulis untuk belajar.
		
	\item Bapak I Wayan Mustika, S.T., M.Eng., Ph.D. selaku dosen pembimbing utama, dan Bapak Ir. Agus Bejo, S.T., M.Eng., D.Eng., IPM. selaku dosen pembimbing pendamping,yang telah dengan penuh kesabaran dan ketulusan memberikan ilmu dan bimbingan terbaik kepada penulis.
	
	\item Seluruh Dosen Program Studi S1 Teknik Elektro Fakultas Teknik Universitas Gadjah Mada yang telah memberikan bekal ilmu kepada penulis.
	
	\item Dhesta dan Ocid selaku teman satu bimbingan yang menjadi teman diskusi selama penelitian.
	
	\item Mas Edwin, Mas Nova, dan teman-teman PT Lunar Inovasi Teknologi yang senantiasa membantu di sisi perangkat keras selama penelitian.
	
	\item Shafira Ulfa Inayati yang menemani mencari judul, memberikan motivasi, tempat berkeluh kesah, dan hingga akhirnya dapat menyelesaikan penelitian.
	
	\item Sahabat penulis cabang Bandung, Leika Kamila, Diva Satrienabilla, Mahesa Rheznindya, Bogel, Alfahan Ilham, M. Syafiqha Alfathihah, dan Adyasa Rafindhra yang telah memberikan semangat, motivasi, dan juga menemani penulis hingga dapat menyelesaikan studi.
	
	\item Sahabat penulis cabang Sleman, Akmalda Seto Triwibowo, Gregorius Paskalis, Cendikia Ishmatuka, Dilia Audi, Michael Ananteo, Syafaat Mahrus, Kurnia Wisesa, Nadia Gustiranda, Rere, dan Rifqi Maulana yang telah banyak memberikan bantuan dan motivasi selama studi.
	
	\item Von Devot, Marco, Ibnu, Haru, Tripod, Ipay, Markonah, dan kucing-kucing UGM lainnya yang menggemaskan.

\end{enumerate}

Penulis menyadari sepenuhnya bahwa laporan skripsi ini masih jauh dari sempurna, untuk itu semua jenis saran, kritik dan masukan yang bersifat membangun sangat penulis harapkan. Akhir kata, semoga tulisan ini dapat memberikan manfaat dan memberikan wawasan tambahan bagi para pembaca dan khususnya bagi penulis sendiri.

\begin{flushright}
	\begin{tabular}{c}
		Yogyakarta, 13 Januari 2023 \\
		\vspace{1cm} \\
		Airlangga Rasyad Fidiyanto
	\end{tabular}
\end{flushright}