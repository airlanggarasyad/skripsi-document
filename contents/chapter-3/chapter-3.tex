\chapter{METODE TUGAS AKHIR}

\section{Alat dan Bahan Tugas Akhir}
Dalam penelitian ini digunakan beberapa perangkat keras dan perangkat lunak. Perangkat keras yang digunakan dalam penelitian ini adalah sebagai berikut
\begin{enumerate}
	\item Apple MacBook Pro dengan prosesor Apple M1 Pro 16-\textit{core} dan memori terpadu 16 GB menggunakan sistem operasi macOS Ventura
	\item STM32 Nucleo-WL55JC1 berbasis ARM Cortex-M0 dan ARM Cortex-M4
	\item Modul GNSS Teseo LIV3FL
	\item Modul antena Taoglas xxxx
	\item Kabel \textit{jumper} untuk purwarupa alat
\end{enumerate}
Selanjutnya, perangkat lunak dan pustaka yang diunakan adalah sebagai berikut
\begin{enumerate}
	\item STM32CubeIDE sebagai IDE pada \textit{end node}
	\item Visual Studio Code sebagai IDE untuk merancang situs jaringan penampil
	\item Amazon Web Service untuk merancang API meliputi yang Lambda, API \textit{Gateway}, dan RDS
	\item MySQL sebagai basis data untuk pengolahan data
	\item Postman untuk menguji API yang telah dirancang
\end{enumerate}

\section{Alur Tugas Akhir}

\section{Studi Literatur}

\section{Analisis Kebutuhan Sistem}

\section{Perancangan \textit{End Node}}

\section{Perancangan API dan Basis Data}

\section{Perancangan Web Penampil}