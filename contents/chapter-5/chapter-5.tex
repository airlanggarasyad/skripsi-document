\chapter{KESIMPULAN DAN SARAN}

\section{Kesimpulan}
Berdasarkan penelitian yang telah dilakukan, dapat ditarik beberapa kesimpulan sebagai berikut:
\begin{enumerate}
	\item Penggunaan multi-constellation pada modul Teseo-LIV3FL dapat memperbaiki performa sistem dalam berbagai kondisi seperti ditunjukan pada pengujian Rapid Static Survey pada skenario basement. Untuk mendapat performa terbaik sistem maka posisi paling idealnya adalah berada di kondisi ruang terbuka dengan halangan seminimal mungkin.
	\item Algoritma mode daya rendah pada modul Teseo-LIV3FL memungkingkan modul untuk bekerja dengan efisien dan dapat digunakan dalam penggunaan jangka panjang. Arus yang mengalir pada modul adalah sebesar 7uA pada mode stand by dan 50mA pada mode akuisisi.
	\item Berdasarkan pengujian secara langsung, firmware dapat melacak posisi Bus dengan baik. Adapun visibilitas satelit yang didapat berada dalam rentang empat s.d. dua belas dan nilai HDOP berada pada rentang 0.4 s.d 2. Meskipun sempat tidak dapat menerima isyarat GNSS di daerah tertentu, sistem dapat melakukan pemulihan dan melanjutkan pelacakan kembali.
\end{enumerate}

\section{Saran}
Adapun saran untuk penelitian lebih lanjut adalah sebagai berikut:
\begin{enumerate}
	\item Perlu dilakukan pengujian lebih lanjut dalam kondisi berbeda seperti kondisi langit tidak ideal ketika cuaca buruk.
	\item Perlu ditambahkan basis data untuk menyimpan data pelacakan dan mekanisme penyimpanan seperti penggunaan REST API.
	\item Untuk mempermudah visualisasi hasil pelacakan Bus dapat dikembangkan sebuah aplikasi yang dapat berjalan di berbagai macam gawai seperti aplikasi berbasis web atau PWA.
\end{enumerate}

