\chapter{PENDAHULUAN}

\section{Latar Belakang}
Salah satu moda transportasi dalam kota paling populer di Indonesia adalah bus. Daerah Istimewa Yogyakarta juga telah menyediakan dua buah layanan bus publik seperti Trans Jogja dan Teman Bus. Salah satu faktor yang membuat penggunaan bus cukup populer adalah cakupan wilayahnya luas dan juga murah \cite{Rohani2013}. Selain itu, lalu lintas yang padat dan lahan parkir yang terbatas juga menjadi motivasi beberapa orang untuk menggunakan transportasi publik. Jika peningkatan jumlah pada suatu daerah sangat tinggi maka dibutuhkan fasilitas transportasi umum yang layak seperti bus \cite{Sutandi2015}.

Pada awal bulan Maret tahun 2022, Rektor Universitas Gadjah Mada, Prof. Ir. Panut Mulyono, M.Eng., D.Eng., meluncurkan dua buah bus listrik untuk transportasi internal kampus. Kedua bus ini adalah inovasi dari UGM untuk memudahkan mobilisasi mahasiswa di area kampus seluas 183.36 hektar ini dan mengurangi penggunaan energi fosil secara bersamaan. Setiap bus akan memutari UGM sebanyak sepuluh kali dengan setiap putaran membutuhkan satu jam. Dengan adanya fasilitas bus kampus Trans Gadjah Mada diharapkan dapat membuat lingkungan kampus menjadi lebih nyaman dan kondusif.

Salah satu masalah yang banyak dikeluhkan oleh civitas akademika UGM adalah ketidakpastian waktu kedatangan Trans Gadjah Mada. Meskipun sudah diberikan jadwal estimasi kedatangan bus, terkadang waktu kedatangan bus tidak sesuai dikarenakan faktor cuaca, lalu lintas di sekitar Jalan Persatuan yang padat, dan faktor lainnya. Oleh karena itu, dibutuhkan suatu sistem untuk melacak posisi dari bus kampus agar calon penumpang dapat mengestimasi kapan bus yang akan ditumpangi datang.

Masalah serupa juga terjadi di India. Berdasarkan  \cite{Sutar2016}, masyarakat India hanya mengetahui waktu kedatangan berdasarkan jadwal saja tanpa mengetahui posisi terbaru dari bus yang akan ditumpangi. Penelitian yang dilakukan oleh \cite{Sneha2014} menunjukan bahwa sistem pelacak berbasis GPS telah diimplementasikan di beberapa negara, tetapi belum diimplementasikan di Indonesia, khususnya di lingkungan Universitas Gadjah Mada.

\textit{Global Positioning System} atau GPS adalah teknologi sistem navigasi berbasis satelit yang dapat menunjukan posisi secara akurat. Dengan adanya sistem pelacak lokasi bus Trans Gadjah Mada berbasis GPS, diharapkan dapat membantu civitas akademika Universitas Gadjah Mada untuk mengestimasi waktu kedatangan bus.

\section{Rumusan Masalah}
\section{Batasan Tugas Akhir}
Batasan masalah pada tugas akhir ini adalah:
\begin{enumerate}
	\item Hanya digunakan satu buah \textit{end node} dan pengambilan data dilakukan pada satu putaran Trans Gadjah Mada dimulai dari Halte 0 (Perpustakaan Universitas Gadjah Mada) sampai kembali ke Halte 0.
	\item Penelitian ini difokuskan pada pengembangan \textit{firmware} pada \textit{platform} STM32.
	\item Aspek perangkat keras \textit{node} dan \textit{gateway} bukan merupakan fokus pada penelitian ini.
\end{enumerate}
\section{Tujuan Tugas Akhir}
\section{Manfaat Tugas Akhir}
\section{Sistematika Penulisan}
\textbf{BAB I}

\textbf{BAB II}

\textbf{BAB III}

\textbf{BAB VI}

\textbf{BAB V}